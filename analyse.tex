% Created 2015-09-28 Mon 09:11
\documentclass[11pt]{article}
\usepackage[utf8]{inputenc}
\usepackage[T1]{fontenc}
\usepackage{fixltx2e}
\usepackage{graphicx}
\usepackage{grffile}
\usepackage{longtable}
\usepackage{wrapfig}
\usepackage{rotating}
\usepackage[normalem]{ulem}
\usepackage{amsmath}
\usepackage{textcomp}
\usepackage{amssymb}
\usepackage{capt-of}
\usepackage{hyperref}
\date{\today}
\title{Analyse III}
\hypersetup{
 pdfauthor={},
 pdftitle={Analyse III},
 pdfkeywords={},
 pdfsubject={},
 pdfcreator={Emacs 24.5.1 (Org mode 8.3.1)}, 
 pdflang={English}}
\begin{document}

\maketitle
\tableofcontents


\section{Chapitre 2}
\label{sec:orgheadline1}
Intégrales curvilignes

\begin{itemize}
\item \uline{Def} Courbes
\end{itemize}
\(\Gamma \subset \mathbb{R}^d\) est une courbe simple si \(\exists I \subset \mathbb{R}\)  (interval) \(\gamma: I \rightarrow \mathbb{R}^d\) : continu. tel que
\begin{enumerate}
\item \(\gamma (I) = \tau\) (courbe)
\item \$\(\gamma\) * Chapitre 2
\end{enumerate}
Intégrales curvilignes

\begin{itemize}
\item \uline{Def} Courbes
\end{itemize}
\(\Gamma \subset \mathbb{R}^d\) est une courbe simple si \(\exists I \subset \mathbb{R}\)  (interval) \(\gamma: I \rightarrow \mathbb{R}^d\) : continu. tel que
\begin{enumerate}
\item \(\gamma (I) = \tau\) (courbe)
\item \(\gamma (t_1) \neq \gamma(t_2)\) \(\forall t_1,t_2 \in I\) (simple)
\end{enumerate}
\(\gamma\) : une paramétrisation.

\begin{itemize}
\item Ex. 1 :
\end{itemize}
\(\Gamma = \{ x \in \mathbb{R}^2 : |x|=1\}\)
dessin cercle \(\gamma: \left[0,2\pi\right] \rightarrow \mathbb{R}^2\) \(\gamma(\theta) = (\cos \theta, \sin \theta)\)
\begin{itemize}
\item Ex. 2 : (helix)
\end{itemize}
dessin helix \(\gamma \mathbb{R} \rightarrow \mathbb{R}^3\) \(\gamma(t) = (\cos t, \sin t,t)\)
\begin{itemize}
\item Ex. 3 :
\end{itemize}
\(\mathbb{R} \rightarrow \mathbb{R}^2\)
\(\gamma(t) = (t^3,t^2)\)
dessin
\begin{itemize}
\item Ex. 4:
\end{itemize}
\(\gamma : \mathbb{R} \rightarrow \mathbb{R}^2\)
\(\gamma(t) = (t,|t|)\)
dessing
\begin{itemize}
\item Ex. 5 :
\end{itemize}
\(\mathbb{R}^2\) dessin courbe pas simple

\begin{itemize}
\item \uline{Def 2} : \(\tau\) : courbe simple est dite fermée si \(\gamma(a) = \gamma(b)\) \(\left(I = \left[a,b\right]\right)\)
\begin{itemize}
\item Ex. 1 : fermé \(\gamma(0)=\gamma(2\pi) = (1,0)\) image patate.
\item Ex. 2,3,4 : non fermé.
\end{itemize}

\item \uline{Def 3}: \(\Gamma\) : courbe est régulière si \(\exists \left[a,b\right], \gamma: \gamma \cdot \left[a,b\right] \rightarrow \mathbb{R}^d\) tel que
\begin{itemize}
\item \(\gamma \in C^1 \left(\left[a,b\right]: \mathbb{R}^2\right)\)
\item \(\gamma ' (t) \neq 0 \in \mathbb{R}^d\) \(\left( \left(\gamma_1'(t),...,\gamma_d'(t) \right) \neq (0,...,0) \right)\)
\end{itemize}

\item Ex.1 : régulière \(\gamma'(t) = (-\sin t, cos t) \neq (0,0)\)
\item Ex.2 : régulière \(\gamma'(y) = (-\sin t, cos t, 1) \neq (0,0,0)\)
\item Ex.3 : \(\gamma' (t) = (3t^2,2t)\) et \(\gamma'(0) = (0,0)\) : \(\Gamma\) : n'est pas régulière.
\item Ex.4 : \(\gamma\) n'est pas diff. en \(t=0\). Donc \(\Gamma\) n'est pas régulière.
\end{itemize}

\uline{Remarque :}
\begin{itemize}
\item \(\Gamma\) : régulière. La ligne tangente en \(\gamma(t_0)\).
\end{itemize}

(L) : \(\gamma(t_0) + \gamma'(t_0)(t-t_0)\)

\begin{itemize}
\item \(\Gamma\) : courbe \(\subset \mathbb{R}^d\)
\end{itemize}

\(f: \Gamma \rightarrow \mathbb{R}\): continue.

\uline{Def :} \(\int\limits_{T} f := \int\limits_{a}^{b} f(\gamma(t)) ||\gamma'(t)|| dt\)

\(\gamma : \left[a,b\right] \rightarrow \mathbb{R}^d\): une paramétrisation de \(\Gamma\).

La longueur de \(\Gamma\) : \(\int\limits_{\Gamma} 1 = \int\limits_{a}^{b}||\gamma'(t)|| dt\)

\begin{itemize}
\item Ex.1 : \(\int\limits_{\Gamma}f\) avec \(f=1\)
\end{itemize}
\(= \int\limits_{0}^{2\pi} ||\gamma'(t)|| dt = \int\limits_{0}^{2\pi} 1 dt = 2\pi\)

\begin{itemize}
\item Ex. : \begin{align*}
\end{itemize}
\(\int\)\limits\(_{\Gamma}\) : \\
f(x,y) = \sqrt{x^2+4y^2} \\
\(\Gamma\) = \left\lbrace (x,y) \(\in\) \mathbb{R}\(^{\text{2}}\); 2y = x\(^{\text{2}}\); x \(\in\) \left[0,1\right]\right\rbrace \\
\(\gamma\) : \left[0,1\right] \(\rightarrow\) \mathbb{R}\(^{\text{2}}\)\\
t \(\rightarrow\) (t,\frac{t^2}{2}\\
\(\int\)\limits\(_{\Gamma}\) f \&= \(\int\)\limits\(_{\text{0}}^{\text{1}}\) f(\(\gamma\)(t)) || \(\gamma\)'(t)|| dt\\
\&= \(\int\)\limits\(_{\text{0}}^{\text{1}}\) \sqrt\{t\(^{\text{2}}\) + 4 \frac{t^4}{4}\} \sqrt{1 + t^2} dt\\
\&= \(\int\)\limits\(_{\text{0}}^{\text{1}}\) \sqrt{t^2 + t^4} \sqrt{1 + t^2} dt\\
\&= \(\int\)\limits\(_{\text{0}}^{\text{1}}\) t (1+t\(^{\text{2}}\)) dt = \frac{t^2}{2} + \frac{t^4}{4} \big|\(_{\text{0}}^{\text{1}}\)= \frac{3}{4} 
\end{align*}

image courbe

\begin{align*}
\int\limits_{\Gamma} \simeq \sum\limits_i |\Gamma_i| f(\gamma(t_i))\\
\gamma(t_i) \in \Gamma_i \\
\simeq \sum\limits_i |\Gamma_i| f(\gamma(t_i))\\
\end{align*}
\(\Gamma_i = \gamma \left( \left[ t_i, t_{i+1} \right] \right)\): \(\gamma\) une paramétrisation \(\left[a,b\right] \rightarrow \mathbb{R}^d\)

Donc \(\int\limits_{\Gamma} \simeq \sum\limits ||\gamma(t_i)'|| (t_{i+1} - t_i) f(\gamma(t_i)) \simeq \int\limits_{a}^{b} ||\gamma'(t)|| f(\gamma(t)) dt\)(t\(_{\text{1}}\)) \(\neq\) \(\gamma\)(t\(_{\text{2}}\))\$ \(\forall t_1,t_2 \in I\) (simple)
\(\gamma\) : une paramétrisation.

\begin{itemize}
\item Ex. 1 :
\end{itemize}
\(\Gamma = \{ x \in \mathbb{R}^2 : |x|=1\}\)
dessin cercle \(\gamma: \left[0,2\pi\right] \rightarrow \mathbb{R}^2\) \(\gamma(\theta) = (\cos \theta, \sin \theta)\)
\begin{itemize}
\item Ex. 2 : (helix)
\end{itemize}
dessin helix \(\gamma \mathbb{R} \rightarrow \mathbb{R}^3\) \(\gamma(t) = (\cos t, \sin t,t)\)
\begin{itemize}
\item Ex. 3 :
\end{itemize}
\(\mathbb{R} \rightarrow \mathbb{R}^2\)
\(\gamma(t) = (t^3,t^2)\)
dessin
\begin{itemize}
\item Ex. 4:
\end{itemize}
\(\gamma : \mathbb{R} \rightarrow \mathbb{R}^2\)
\(\gamma(t) = (t,|t|)\)
dessing
\begin{itemize}
\item Ex. 5 :
\end{itemize}
\(\mathbb{R}^2\) dessin courbe pas simple

\begin{itemize}
\item \uline{Def 2} : \(\tau\) : courbe simple est dite fermée si \(\gamma(a) = \gamma(b)\) \(\left(I = \left[a,b\right]\right)\)
\begin{itemize}
\item Ex. 1 : fermé \(\gamma(0)=\gamma(2\pi) = (1,0)\) image patate.
\item Ex. 2,3,4 : non fermé.
\end{itemize}

\item \uline{Def 3} : \(\Gamma\) : courbe est régulière si \(\exists \left[a,b\right], \gamma: \gamma \cdot \left[a,b\right] \rightarrow \mathbb{R}^d\) tel que
\begin{itemize}
\item \(\gamma \in C^1 \left(\left[a,b\right]: \mathbb{R}^2\right)\)
\item \(\gamma ' (t) \neq 0 \in \mathbb{R}^d\) \(\left( \left(\gamma_1'(t),...,\gamma_d'(t) \right) \neq (0,...,0) \right)\)
\end{itemize}

\item Ex.1 : régulière \(\gamma'(t) = (-\sin t, cos t) \neq (0,0)\)
\item Ex.2 : régulière \(\gamma'(y) = (-\sin t, cos t, 1) \neq (0,0,0)\)
\item Ex.3 : \(\gamma' (t) = (3t^2,2t)\) et \(\gamma'(0) = (0,0)\) : \(\Gamma\) : n'est pas régulière.
\item Ex.4 : \(\gamma\) n'est pas diff. en \(t=0\). Donc \(\Gamma\) n'est pas régulière.
\end{itemize}

\uline{Remarque :} 
\begin{itemize}
\item \(\Gamma\) : régulière. La ligne tangente en \(\gamma(t_0)\).
\end{itemize}

(L) : \(\gamma(t_0) + \gamma'(t_0)(t-t_0)\)

\begin{itemize}
\item \(\Gamma\) : courbe \(\subset \mathbb{R}^d\)
\end{itemize}

\(f: \Gamma \rightarrow \mathbb{R}\): continue.

\uline{Def :} \(\int\limits_{T} f := \int\limits_{a}^{b} f(\gamma(t)) ||\gamma'(t)|| dt\)

\(\gamma : \left[a,b\right] \rightarrow \mathbb{R}^d\): une paramétrisation de \(\Gamma\).

La longueur de \(\Gamma\) : \(\int\limits_{\Gamma} 1 = \int\limits_{a}^{b}||\gamma'(t)|| dt\)

\begin{itemize}
\item Ex.1 : \(\int\limits_{\Gamma}f\) avec \(f=1\)
\end{itemize}
\(= \int\limits_{0}^{2\pi} ||\gamma'(t)|| dt = \int\limits_{0}^{2\pi} 1 dt = 2\pi\)

\begin{itemize}
\item Ex. : \begin{align*}
\end{itemize}
\(\int\)\limits\(_{\Gamma}\) : \\
f(x,y) = \sqrt{x^2+4y^2} \\
\(\Gamma\) = \left\lbrace (x,y) \(\in\) \mathbb{R}\(^{\text{2}}\); 2y = x\(^{\text{2}}\); x \(\in\) \left[0,1\right]\right\rbrace \\
\(\gamma\) : \left[0,1\right] \(\rightarrow\) \mathbb{R}\(^{\text{2}}\)\\
t \(\rightarrow\) (t,\frac{t^2}{2}\\
\(\int\)\limits\(_{\Gamma}\) f \&= \(\int\)\limits\(_{\text{0}}^{\text{1}}\) f(\(\gamma\)(t)) || \(\gamma\)'(t)|| dt\\
\&= \(\int\)\limits\(_{\text{0}}^{\text{1}}\) \sqrt\{t\(^{\text{2}}\) + 4 \frac{t^4}{4}\} \sqrt{1 + t^2} dt\\
\&= \(\int\)\limits\(_{\text{0}}^{\text{1}}\) \sqrt{t^2 + t^4} \sqrt{1 + t^2} dt\\
\&= \(\int\)\limits\(_{\text{0}}^{\text{1}}\) t (1+t\(^{\text{2}}\)) dt = \frac{t^2}{2} + \frac{t^4}{4} \big|\(_{\text{0}}^{\text{1}}\)= \frac{3}{4} 
\end{align*}

image courbe

\begin{align*}
\int\limits_{\Gamma} \simeq \sum\limits_i |\Gamma_i| f(\gamma(t_i))\\
\gamma(t_i) \in \Gamma_i \\
\simeq \sum\limits_i |\Gamma_i| f(\gamma(t_i))\\
\end{align*}
\(\Gamma_i = \gamma \left( \left[ t_i, t_{i+1} \right] \right)\): \(\gamma\) une paramétrisation \(\left[a,b\right] \rightarrow \mathbb{R}^d\)

Donc \(\int\limits_{\Gamma} \simeq \sum\limits ||\gamma(t_i)'|| (t_{i+1} - t_i) f(\gamma(t_i)) \simeq \int\limits_{a}^{b} ||\gamma'(t)|| f(\gamma(t)) dt\)
\end{document}
